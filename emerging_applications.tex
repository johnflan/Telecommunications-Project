
\section{Emerging  Applications}

\subsection{Entertainment}

In current times VoIP is reaching its capabilities as alternative to traditional telephone. Slowly software industry is trying to integrate VoIP protocol into their products. Currently integrated VoIP is used to enhance the user experience by allowing for intuitive and direct way of communicating with fellow users. And the biggest beneficent of this process is gaming industry. 

As high speed Internet connection became widely spread and used, gaming industry began to focus on creating multiplayer games. Users playing early multiplayer games did not have a chance to communicate verbosely; instead they were forced to write short messages through chat boxes. Introduction of VoIP allowed to overcome slow and cumbersome communication and created additional way of enhancing the game play\cite{website:voip_impact_on_gaming}.

However early implementation of VoIP in multiplayer games were far from perfect. Jitter, lost packets, microphone echo effect, turning off game resulting in termination of voice chat were amongst main complains from players. As a solution third party companies amongst them: Mumble, Ventrilo and Teamspeak, come up with more advanced implementation of VoIP for players. Allowing them to create their teams, adjust max input output volumes, regulate voice transmission thresholds and/or echo cancelation. 

But again there was a drawback to this solution(s). To manage you voice communication players were forced to minimise game first prior to configuring or adjusting voice chat options.

Currently there is not golden solution to the above solutions and users are forced to choose either functionality or ease of use.
Gaming market changes rapidly though and there might be possibility that VoIP services could be both reliable, user-friendly and be integrated in all of games. There are two main games markets: consoles and PC. First let's discuss consoles, as this is the fastest growing gaming sector\cite{website:shift_in_gaming_demographics}.

Consoles in its nature of unified platform, can provide an integrated VoIP services and expose it to games developers so they have an option of utilising consoles VoIP implementation in their products. As console market is very competitive all main console companies: Sony, Microsoft, Nintendo are trying to develop good quality integrated voice services to gain a competitive advantage\cite{website:voip_adataion}.

PC as common platform is not so easy to consolidate as many different OS (operating systems) can be used by users. Building integrated VoIP for all OS, old and new would be unfeasible. However in 2004 Valve Corporation released a platform called Steam. It provides an easy way to buy and store your games online. It also has an exposed API that provides a wrapper for any game it sales. Wrapper allows users for accessing some of Steam�s resources like friend�s lists and groups or instant messaging. Steam grew popular over last few years and became a predominant games distributor for PC (estimated 70\% market share\cite{steam_market_share}). In 2007 Valve implemented a VoIP service. It's far from being perfect but lack of treating competitor does not stimulate Valve to further work on enhancing its VoIP implementation. Although Steam provides users with a clean interface, abilities to create "chat rooms" and manage transmission/receive volumes.

All of the platforms however were faced with one big problem how to serve large client base without large dedicated server base. In 2010 Steam reached 30 \cite{website:steam_growth}, Xbox Live 23\cite{xbox_live_market}, Playstation3 33.5 \cite{website:ps3_marketshare} millions users. To provide reliable VoIP services all platforms similary to Skype are using peer-to-peer Voice-Over-IP clients. That way service providers are distributing workload to each client and only controlling how service is provided (authorisation, discovery of peers).

\subsection{Smart-phone Integration}

After establishing a solid foundation in PC market VoIP took a natural step and start spreading to mobile devices. Data plans are now a common option for contract and pre-paid mobile users. Combined with widespread implementation of 3G (3rd generation mobile services), mobiles are equipped with couple of channels that VoIP can take an advantage of to allow mobile users for cheaper calls via VoIP protocol. There are numbers of available solutions for mobiles mainly in the smartphones segment.

However VoIP implementations are still suffering from many issues: packet loss and delay, jitter, acoustic echo and OS tuning. Despite that the biggest thing that is crippling VoIP is the mobile companies themselves. Apple's iPhone has restricted VoIP to only Wi-Fi networks, blocking 3G's data channel as means of VoIP communication, this of course limits where and when user can use VoIP. Microsoft with its Windows Mobile also limits capabilities of VoIP, in contrary to iPhone it allows VoIP to access 3G but it limits access to internal handset earpiece. Resulting in bad quality of received voice, especially if external speaker is located on side or back of phone\cite{website:voip_cripling_by_apple}.

There is no definite statement by mobile producers why such a actions were taken. However one commonly shared opinion is that telecom companies which are closely cooperating with mobile producers are lobbing for reducing the capabilities of VoIP in mobiles. To protect their market and force mobile customers to use only their services. This state of matters would stay like that but a new alternative OS appeared on market � Android by Google. Android is gaining popularity as it is open-source product as it is not restricting what device can or can�t do. That means no restrictions for any implementation developed for Android. As this alternative is gaining more and more mobile market share, proprietary OS�s will have to drop their restrictions on VoIP to stay competitive.

Other issue that can really affect mobile VoIP is lack of bandwidth. If bandwidth pipe is too small for transmitting VoIP packets quality of conversation will suffer. This is a real issue if VoIP is used on Wi-Fi network which is also used by other users. Things are looking better in Vo3G (Voice over 3G), but there are still few concerns. Currently there are several concurrent applications running on users' mobile phones and few of them require bandwidth to work. That is why 3G bandwidth can be an issue for a mobile VoIP user. The ideal bandwidth that 3G can reach is 2.4 Mb/s, however realistically a stationary user can expect to reach 2 Mb/s. That is enough for most of application running alongside VoIP application. But the faster user is moving the smaller bandwidth get. Pedestrian can expect 384Kb/s, vehicle passenger only 144Kb/s \cite{website:3g_spec}. 

Although most of VoIP codecs require less than 40 Kb/s and this is enough to support quality VoIP service\cite{website:voip_codecs}, but this shows that bandwidth can still be an issue if mobile user is running many application and happens to be in position where bandwidth is greatly reduced.


\subsection{VoIP and Video Conferencing}
Every business is looking to reduce cost of operating and make its employees work more effectively. On top of that many companies have their branches spread across the whole world. VoIP protocol proved to be very successive as cheap method of bringing people from across the world together. But this was not enough for businesses who demanded a new easier and more direct way for its employees to communicate and cooperate. That is why video conferencing (combination of video and VoIP services) became very popular across all global businesses. But there are issues and tradeoffs tied to this approach, ranging from technical to psychological. Usage of video conferencing software requires some ports (depending on what VoIP standard was used) to be open and since the connection to other user (node) is based on p2p protocol, there is a possibility of a malicious actions taken by outside third party. H.323 protocol for example was not designed with security as its prime feature; many open ports increase a chance of attack on system\cite{website:vid_conf_voice_protocol_overview}.

Usually video conferencing is going to take place in offices however mobile video conferencing has already been implemented. As mentioned in section above VoIP alone needs as much as 40Kb/s of bandwidth, now we need to increase bandwidth requirements by the amount of bandwidth needed by video transmission. Naturally the bigger and better quality images are being sent/received the more bandwidth will be required. 
To psychological problems with video conferencing we can account the discomfort in delays of responses. Delays as small as 300ms can be spotted by users and create a feeling of discomfort and bad communication. Another issue is related to facial expressions, because of hardware limitations users are looking at the monitor rather than at the camera creating bad image in eye of the other person as it looks like person is avoiding an eye contact\cite{website:vid_conf_overview}.


