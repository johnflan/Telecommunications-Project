
\section{Emerging  Applications}

\subsection{Entertainment}

In current times VoIP is reaching its capabilities as purely telephone alternative. Slowly entrepreneurs are trying to integrate VoIP protocol into their products. Currently integrated VoIP is used to enhance the user experience with given application. And the biggest beneficent of this process is gaming industry. 

As high speed Internet connection became widely used, gaming industry began to focus on creating multiplayer games. Users playing early multiplayer games did not have a chance to communicate verbosely; instead they were forced to write short messages through chat boxes. Introduction of VoIP allowed to overcome slow communication and created additional way of enhancing the game play\cite{website:voip_impact_on_gaming}.


However early implementation of VoIP in multiplayer games were far from perfect. Jitter, lost packets, microphone echo effect, turning off game terminates voice chat were amongst main complains from players. As a solution third party companies  amongst them: Mumble, Ventrilo or Teamspeak, come up with more advanced implementation of VoIP for players, allowing them to create their teams, adjust max input output volumes, regulate voice transmission thresholds and/or echo cancelation\cite{website:mumble_faq}. 

But again there was a drawback to this solution(s). To manage you voice communication players were forced to minimise game first prior to configuring voice chat.

Currently there is not golden solution to the above solutions and users are forced to choose either functionality or ease of use.
Gaming market changes rapidly though and there might be possibility that VoIP services could be both reliable, user-friendly and be integrated in all of games. There are two main games markets: consoles and PC. First let's discuss consoles, as this is the fastest growing sector\cite{website:shift_in_gaming_demographics}.

Consoles in its nature can provide an integrated VoIP services and expose it to games developers through an API. As console market is very competitive all main console companies: Sony (Playstation3 Jajah\cite{website:playstation3_voip_figures}), Microsoft (Xbox's Live)\cite{website:xbox_figures}, Nintendo (Wii Speak)\cite{website:wii_voip} are trying to develop good quality integrated voice services to gain a competitive advantage\cite{website:voip_adataion}.

PC as platform is not so easy to consolidate, however in 2004 Valve Corporation released a platform called Steam. Steam provides an easy way to buy and store your games online. It also has an API to provide a wrapper for any game it sales. On top of that Steam allows users to create a friends list and groups. Steam grew popular over last few years and became a predominant games distributor for PC (estimated 70\% market share \cite{steam_market_share}). In 2007 Valve implemented a VoIP service. It's far from being perfect and lack of treating competitor does not stimulate Valve to further work on its VoIP implementation, but it provides users with a clean interface, abilities to create "chat rooms" and manage transmission/receive volumes.

All of the platforms however were faced with one big problem how to serve large client base without large dedicated server base. In 2010 Steam reached 30  \cite{website:steam_growth}, Xbox Live 23  \cite{xbox_live_market}, Playstation3 33.5 \cite{website:ps3_marketshare} million users. To provide reliable VoIP services all platforms just like Skype are using peer-to-peer Voice-Over-IP clients. That way service providers are distributing workload to each client and only controlling how service is provided (authorisation, discovery of peers).

\subsection{Smart-phone Integration}

After establishing a solid foundation in PC market VoIP took a natural step and start spreading to mobile devices. Data plans are now a common option for contract and pre-paid mobile users. Combined with widespread implementation of 3G (3rd generation mobile services), mobiles are equipped with couple of channels that VoIP can take an advantage of to allow mobile users for cheaper calls via VoIP protocol. There are numbers of available solutions for mobiles mainly in the smartphones segment.

However VoIP implementations are still suffering from many issues: packet loss and delay, jitter, acoustic echo and OS tuning. However the biggest thing that is crippling VoIP is the mobile companies themselves. Infamous Apple's iPhone has restricted VoIP to only Wi-Fi networks, blocking data channel as means of VoIP communication, this of course limits where and when user can use VoIP. Microsoft with its Windows Mobile also limits capabilities of VoIP, in contrary to iPhone it allows VoIP to access 3G but it limits access to internal handset earpiece. Resulting in bad quality of received voice, especially if external speaker is located on side or back of phone\cite{website:voip_cripling_by_apple}.

There is no definite statement by mobile producers why such a actions were taken. However one commonly shared opinion is that telecom companies which are closely cooperating with mobile producers. Telecom companies are lobbing for reducing capabilities of VoIP in mobiles, to protect their market and force mobile customers to use only their services. Thank to open-source an Android OS was released and now it is gaining more and more attention. Because of the nature of Android telecom companies are not able to pressure developers to restrict VoIP implementations. As Android is gaining larger market-share, proprietary companies would be forced to allow for fair VoIP implementations to stay competitive.

Other issue that can really affect mobile VoIP is lack of bandwidth. If bandwidth pipe is too small for transmitting VoIP packets quality of conversation will suffer. This is a real issue if VoIP is used on Wi-Fi network which is also used by other users. Things are looking better in Vo3G (Voice over 3G), but there are still few concerns. Currently there are several concurrent applications running on users' mobile phones and few of them require bandwidth to work. That is why 3G bandwidth can be an issue for a mobile VoIP user. The ideal bandwidth that 3G can reach is 2.4 Mb/s, however realistically a stationary user can expect to reach 2 Mb/s. That is enough for most of application running alongside VoIP application. But the faster user is moving the smaller bandwidth get. Pedestrian can expect 384Kb/s, vehicle passenger only 144Kb/s \cite{website:3g_spec}. 

Although most of VoIP codecs require less than 40 Kb/s and this is enough to support quality VoIP service\cite{website:voip_codecs}, but this shows that bandwidth can still be an issue if mobile user is running many application and happens to be in position where bandwidth is greatly reduced.


\subsection{VoIP and Video Conferencing}
Every business is looking to reduce cost of operating and make its employees work more effectively. On top of that many companies have their branches spread across the whole world. VoIP protocol proved to be very successive as cheap method of bringing people from across the world together. But this was not enough for businesses who demanded a new easier and more direct way for its employees to cooperate. That is why video conferencing (combination of video and VoIP services) became very popular across all global businesses. But there are issues and tradeoffs tie to this approach ranging from technical to psychological. 
Usage of video conferencing software requires some ports (depending on standard used) to be open and since the connection to other user (node) is based on p2p protocol, there is a possibility of a malicious actions taken by outside third party. H.323 protocol for example was not designed with security as its prime feature; many open ports increase a chance of attack on system\cite{website:vid_conf_voice_protocol_overview}.

Usually video conferencing is going to take place in offices however mobile video conferencing has been implemented. As mentioned in section above VoIP alone needs as much as 40Kb/s of bandwidth, now we need to increase he bandwidth requirements by the video. Naturally the bigger and better quality images are being sent/received the more bandwidth will be required.
To psychological problems with video conferencing we can account the discomfort in delays of responses. Delays as small as 300ms can be spotted by users and create a feeling of discomfort and bad communication. Another issue is related to facial expressions, because of hardware limitations users are looking at the monitor rather than at the camera creating bad image in eye of other person as it looks like person is avoiding an eye contact\cite{website:vid_conf_overview}.

\subsection{Unified Communications}

An underlying theme in the development of VoIP technologies is the integration with existing non-real-time forms of communication such as email, voicemail and fax services.

Initially focused at business customers unified communications can provide potentially great value when integrated with business processes, enabling decision makers to be more responsive to real-time data from production and work-flow systems no matter the persons location. The combining of presence information, messaging, conferencing, collaboration, email, calendaring and business systems with IP technology and the plethora of new form factor devices becomes a very powerful concept and provides new opportunities to communication system providers. A number of major firms are competing currently in this area namely Microsoft, Cisco and IBM - with each firm attempting to merge it with their existing platforms.

The concept of unified communications is also making its way slowly into the domestic market, an early leader in this area was RIM with their Blackberry device which suddenly enabled mobile phone users to receive emails to their mobile device instantly. Recently with the increased adoption of IP on the mobile handset users are becoming familiar with sending and receiving Twitter messages, mobile email, uploading photos to social networks. While these communication streams have not yet be integrated they pave the way to having your presence and communications available on whatever device is appropriate at the time, with the network dynamically routing messages to the relevant location.

Recently Google has attempted to position itself as a unified communication provider with a new product called Google Voice. They offer integration of email, sms, traditional mobile phone numbers, VoIP, conferencing, switching calls to other devices mid conversation and numerous other features to users in both a desktop and mobile client.

This area is still in its infancy, but with the increasing adoption of capable IP connected mobile handsets and the adoption cloud based services it appears to be a logical step forward.


