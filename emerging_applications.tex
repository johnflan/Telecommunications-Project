
\section{Emerging  Applications}

\subsection{Entertainment}

In current times VoIP is reaching its capabilities as alternative to the traditional telephone service. Slowly the software industry is trying to integrate the VoIP protocol into their products. Currently integrated VoIP is used to enhance the user experience by allowing for intuitive and direct way of communicating with fellow users. And the biggest beneficent of this process is gaming industry. 

As high speed Internet connection became wide spread, the gaming industry began to focus on creating multiplayer games. Users playing early multiplayer games did not have a chance to communicate verbally; instead they were forced to write short messages through chat boxes. The introduction of VoIP allowed to overcome slow and cumbersome communication and created additional way of enhancing the game play\cite{website:voip_impact_on_gaming}.

However the early implementations of VoIP in multiplayer games were far from perfect. Jitter, lost packets, microphone echo effect, exiting from the game environment which resulted in the termination of voice chat were amongst main complaints from players. As a solution third party companies: Mumble, Ventrilo and Teamspeak, came up with more advanced implementations of VoIP for games. Allowing them to create their teams, adjust max input output volumes, regulate voice transmission thresholds and/or echo cancelation. But again there was a drawback to these solutions, to manage your voice communication players were forced to minimise the game prior to adjusting voice chat options.

Currently there is no golden solution to the above problems and gamers are forced to choose either functionality or ease of use. The gaming market changes rapidly though and there might be a possibility that VoIP services could be reliable, user-friendly and be integrated in all of games. There are two main gamimg platforms: consoles and the traditional PC.

First let's discuss consoles, as this is the fastest growing gaming sector\cite{website:shift_in_gaming_demographics}. Consoles by nature are a unified platform, can provide an integrated VoIP service and expose it to game developers so they have an option of utilising that console's VoIP implementation in their product. As the console market is very competitive all the main console companies: Sony, Microsoft and Nintendo are trying to develop good quality integrated voice services to gain a competitive advantage\cite{website:voip_adataion}.

The PC as a common platform is not so easy to consolidate as many different operating systems exist side-by-side. Building an integrated VoIP solution for all OS, old and new would be unfeasible. However in 2004 Valve Corporation released a platform called Steam. It provides an easy way to buy and store your games online. It also has an exposed API that provides a wrapper for any game it sales. Wrapper allows users to access Steam's resources such as friend's lists, groups or instant messaging. Steam grew popular over the last few years and became a predominant games distributor for the PC platform (estimated 70\% market share\cite{steam_market_share}). In 2007 Valve implemented a VoIP service. It's far from being perfect but the lack of a competitor does not stimulate Valve to further work on enhancing its implementation. Although Steam provides users with a clean interface, abilities to create "chat rooms" and manage transmission/receive volumes.

All of the platforms however were faced with one big problem how to serve a large client base without a large dedicated server farm and large bandwidth demand. In 2010 Steam reached 30 million users \cite{website:steam_growth}, Xbox Live 23 million users\cite{xbox_live_market} and Playstation3 33.5 million users\cite{website:ps3_marketshare}. To provide reliable VoIP services the platforms (similary to Skype) use a peer-to-peer architecture for providing VoIP. That way service providers are distributing workload to each client and only controlling how service is provided via authorisation and discovery services.

\subsection{Smart-phone Integration}

After establishing a solid foundation in the PC market VoIP took a natural step and began spreading to mobile devices. Data plans are now a common option for contract and pre-paid mobile users. Combined with a widespread 3G (3rd generation) mobile data network, handsets are equipped with fast data access that VoIP clients may take advantage of in order to allow mobile users gain cheaper calls.

However VoIP implementations are still suffering from many issues namely packet loss, delay and jitter. Despite that the biggest thing crippling VoIP is the handset manufacturers companies themselves. Apple's iPhone has restricted to VoIP transmission only on Wi-Fi networks, blocking 3G's data channel as means of VoIP communication, this of course limits where and when user can use VoIP. Microsoft with its Windows Mobile also limits the capabilities of VoIP, in contrary to iPhone it allows VoIP to access 3G but it limits access to internal handset earpiece. Resulting in bad quality of received voice, especially if external speaker is located on side or back of phone\cite{website:voip_cripling_by_apple}.

There is no definite statement by mobile producers why such a actions were taken. However one commonly shared opinion is that telecom companies which are closely cooperating with mobile producers to reduce the capabilities of VoIP in mobiles. To protect their market and force mobile customers to use only their services. This state of affairs would be maintined but the introduction of a new OS by Google is set to upset the market dynamics. Android is gaining popularity as it is an open-source product as it does not restrict the functionality of the device. As this alternative is gaining more and more mobile market share, proprietary OS's will have to remove their restrictions on VoIP to stay competitive.

Other issue that can affect mobile VoIP is the lack of bandwidth. If the bandwidth pipe is too small for continuous packet transmission the audio quality of the conversation will be degraded. If UDP packets were used to encode voice then call receiver would hear noticeable gaps in conversation. If TCP were used than all packets that did not made through would have to be retransmitted causing pipe to be overblown with packets. This could possibly be a real issue; however any standard Wi-Fi network has bandwidth large enough to accommodate for number of clients using VoIP services. Bandwidth in Vo3G (Voice over 3G), is usually smaller to Wi-Fi but it is capable of caring a VoIP conversation. The designed/specified bandwidth that 3G can reach is 2.4 Mb/s, however realistically a stationary user can expect to reach 2 Mb/s. Pedestrian can expect 384Kb/s, while a vehicle passenger only 144Kb/s \cite{website:3g_spec}. Most of VoIP codecs require less than 40 Kb/s to sustain reasonable level of service quality\cite{website:voip_codecs}, and it confirms that currently bandwidth should not be a concern affecting VoIP services.

\subsection{VoIP and Video Conferencing}
Every business is looking to reduce the cost of operating and encourage its employees work more effectively. On top of that many companies have their branches spread across the whole world. VoIP has proven to be very successful as a cheap method of bringing people from across the world together. But this was not enough for businesses who demanded a easier and more direct way for its employees to communicate. That is why high quality video conferencing is becoming quite popular across geographically diverse organisations. But there are issues and tradeoffs tied to this solution, ranging from technical to psychological\cite{website:vid_conf_voice_protocol_overview}.

To psychological problems with video conferencing we can account the discomfort in delays of responses. Delays as small as 300ms can be spotted by users and create a feeling of discomfort and bad communication. Another issue is related to facial expressions, because of hardware limitations users are looking at the monitor rather than at the camera creating bad image in eye of the other person as it looks like person is avoiding an eye contact\cite{website:vid_conf_overview}.
Although video conferencing is used widely in businesses, it is also popular method of communication for domestic users, as it allows for more direct and natural way to talk to other people. Video conversations provides means of expressing emotions easier (especially facial or gestures) that cannot be expressed through traditional phone. 

Tandberg is a telecommunication company that in 2009 released a video conferencing phone with LCD screen and camera. This is an attempt to create a new generation of stationary phones that allow for quick and more direct video calls. This product was lunched to target offices, however if this concept is going to adopted, this concept might also be brought to domestic use.

\subsection{Unified Communications}

An underlying theme in the development of VoIP technologies is the integration with existing non-real-time forms of communication such as email, voicemail and fax services.

Initially focused at business customers unified communications can provide potentially great value when integrated with business processes, enabling decision makers to be more responsive to real-time data from production and work-flow systems no matter the persons location. The combining of presence information, messaging, conferencing, collaboration, email, calendaring and business systems with IP technology and the plethora of new form factor devices becomes a very powerful concept and provides new opportunities to communication system providers. A number of major firms are competing currently in this area namely Microsoft, Cisco, Google and IBM - with each firm attempting to merge it with their existing platforms.

The concept of unified communications is also making its way slowly into the domestic market, an early leader in this area was RIM with their Blackberry device which enabled mobile phone users to receive emails to their mobile device instantly. Recently with the increased adoption of IP on the mobile handset users are becoming familiar with sending and receiving Twitter messages, mobile email, uploading photos to social networks. While these communication streams have not yet be integrated they pave the way to having your presence and communications available on whatever device is appropriate at the time, with the network dynamically routing messages to the relevant location.

Recently Google has attempted to position itself as a unified communication provider with a new product called Google Voice. They offer integration of email, sms, traditional mobile phone services, VoIP, conferencing, switching calls to other devices mid conversation and numerous other features to users in both a desktop and mobile client.\cite{website:cellularNewsGoogleUC}

This area is still in its infancy, but with the increasing adoption of capable IP connected mobile handsets and the adoption cloud based services it appears to be a logical step forward.


