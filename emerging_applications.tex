
\section{Emerging  Applications}

\subsection{Entertainment}

In current times VoIP is reaching its capabilities as purely telephone alternative. Slowly entrepreneurs are trying to integrate VoIP protocol into their products. Currently integrated VoIP is used to enhance the user experience with given application. And the biggest beneficent of this process is gaming industry. 

As high speed Internet connection became widely used, gaming industry began to focus on creating multiplayer games. Users playing early multiplayer games did not have a chance to communicate verbosely; instead they were forced to write short messages through chat boxes. Introduction of VoIP allowed to overcome slow communication and created additional way of enhancing the game play\cite{website:voip_impact_on_gaming}.


However early implementation of VoIP in multiplayer games were far from perfect. Jitter, lost packets, microphone echo effect, turning off game terminates voice chat were amongst main complains from players. As a solution third party companies  amongst them: Mumble, Ventrilo or Teamspeak, come up with more advanced implementation of VoIP for players, allowing them to create their teams, adjust max input output volumes, regulate voice transmission thresholds and/or echo cancelation\cite{website:mumble_faq}. 

But again there was a drawback to this solution(s). To manage you voice communication players were forced to minimise game first prior to configuring voice chat.

Currently there is not golden solution to the above solutions and users are forced to choose either functionality or ease of use.
Gaming market changes rapidly though and there might be possibility that VoIP services could be both reliable, user-friendly and be integrated in all of games. There are two main games markets: consoles and PC. First let's discuss consoles, as this is the fastest growing sector\cite{website:shift_in_gaming_demographics}.

Consoles in its nature can provide an integrated VoIP services and expose it to games developers through an API. As console market is very competitive all main console companies: Sony (Playstation3 Jajah\cite{website:playstation3_voip_figures}), Microsoft (Xbox's Live)\cite{website:xbox_figures}, Nintendo (Wii Speak)\cite{website:wii_voip} are trying to develop good quality integrated voice services to gain a competitive advantage\cite{website:voip_adataion}.

PC as platform is not so easy to consolidate, however in 2004 Valve Corporation released a platform called Steam. Steam provides an easy way to buy and store your games online. It also has an API to provide a wrapper for any game it sales. On top of that Steam allows users to create a friends list and groups. Steam grew popular over last few years and became a predominant games distributor for PC (estimated 70\% market share \cite{steam_market_share}). In 2007 Valve implemented a VoIP service. It's far from being perfect and lack of treating competitor does not stimulate Valve to further work on its VoIP implementation, but it provides users with a clean interface, abilities to create "chat rooms" and manage transmission/receive volumes.

All of the platforms however were faced with one big problem how to serve large client base without large dedicated server base. In 2010 Steam reached 30  \cite{website:steam_growth}, Xbox Live 23  \cite{xbox_live_market}, Playstation3 33.5 \cite{website:ps3_marketshare} million users. To provide reliable VoIP services all platforms just like Skype are using peer-to-peer Voice-Over-IP clients. That way service providers are distributing workload to each client and only controlling how service is provided (authorisation, discovery of peers).

\subsection{Smart-phone Integration}

\subsection{VoIP and Video Conferencing}
