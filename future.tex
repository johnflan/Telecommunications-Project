\section{Future}

Attempting to predict the future of a particular technology is a notoriously hazardous pass-time, with the potential of a new previously unseen disruptive technology being just on the horizon. But ignoring this we will attempt to identify trends for VoIP and beyond in this section. We say VoIP and beyond because each of the major standards H.323, SIP and although not a public standard Skype, treat voice traffic as an equal to text, data and video. This progression to ‘unified communications’ has already begun in the corporate domain and as pointed out earlier in this document the early steps are already being made for domestic users.

\subsection{Follow me communication}
At this point is quite clear to see a trend towards mobile connected devices and the provision of cloud based internet services. With more and more of our data being stored in the cloud, the era of the internet terminal may not be so far away. With such devices a user may just pick up a terminal at random and login, gaining access to all their data and services through these online services.
Today VoIP standards allow for a call to be initiated on a particular device and without dropping the call move to another device say from a desk phone to mobile phone and from mobile phone to car phone through the course of a single conversation. But in future we will require these protocols to become device and even location aware not only presenting real-time data in suitable format for the device, but prioritising when to notify the user of these messages by looking at calendar information possibly mixed with location coordinates.
This context aware notification and smart transcription/formatting of messages will become very important as time progresses, eventually it will be an essential mechanism for managing a constant torrent of data.

\subsection{Network Changes}
In the last ten years the fixed line telecoms providers have migrated their networks from largely voice based carriers to IP data networks, but mobile telecoms providers have been insulated from this demand for mobile data until recently largely due to mobile data technology lagging behind wired technology. AT\&T in the USA have been on the leading edge of this migration since the introduction of the iPhone in 2007, with many of the customers of the phone being tech savy and regular internet users. Since the launch of the iPhone AT\&T have spend almost \$37 billion dollars on upgrades to the network to increase its capacity.

Over the next number of years the battle between the traditional telecommunications providers and that of the data providers will be key in defining how we gain access to these communication networks. Currently in the USA we are reading about ‘network neutrality’ and how the telecoms providers want to offer a tiered internet experience to their customers, much like the à la carte TV service Sky Digital offer us today. To date mobile network providers have remained silent on this matter, but with mobile data becoming more important than the traditional voice service, how are the network providers going to operate a financially successful business.

\subsection{Costs}
The price of call between two computers using a VoIP implementation is currently free. This is due to the fact that companies like Skype do not handle the communication traffic. They have only several control servers that allow users to set up link between two parties. Currently the only costs a user incurs is that of the internet connection, which is still quite cheap as high speed broadband becomes almost ubiquitous. This creates a threat for telecom companies as their main market is drifting away from standard traditional mean of communication towards IP based communication. This is going to be tough challenge deal with as data plans for mobiles are cheap and calls through Vo3G are cheaper then standard mobile phone companies rates. Vo3G also allows users to avoid extra cost related to calls between different networks. Mobile companies might end up providing only data plans as they wont be able to compete with VoIP. The cost of international calls are going down and they will continue due to the fact that customers are more likely to set up a VoIP call if they believe that its too expensive for them to use conventional phone. The telecom companies have no choice but to keep reducing the prices of calls in hope that customers would consider it as cheap and price it in a way that convenience wins out over the alternatives. However the cost of the services might be reduced to such low level that telocom companies would not break even. This would certainly make them increase the fees for renting their infrastructure. Which means that at the end of the day its going to be the customers who will pay for continually decreasing prices.
