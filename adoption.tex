\section{Adoption}
The consumers of VoIP can be largely broken into three groups domestic consumers, telcos and corporates.

The adoption of this technology in the domestic market has been largely driven by the availability of cheap broadband as well as the ubiquity of personal computers, and this is predicted to increase with mobile internet connected handsets becoming prevalent.

It is difficult to assess the actual number of domestic users as this technology is offered primarily through a number of commercial providers, being integrated into non-traditional telephony platforms such as gaming platforms and finally the use of private private servers. 

According to TeleGeography Research, Skype in 2009 accounted for 12 percent of international calling minutes, a 50 percent increase over 2008\cite{website:skype_usage}. A more interesting trend is that 36 percent of Skype-to-Skype calls were video based\cite{website:skype_usage}. Which indicates that telephony customers are willing to move to a more immerse conversation using video when appropriate and suitably convenient. In addition to this its is not unusual for a Skype user to be informed that more that 20 million users are currently logged into the system. These numbers are quite impressive when you consider that Skype was founded in 2003 and only recently has consumer technology enabled a reasonable quality of service as well as the ability to use this service on a mobile handset.

In the enterprise environment, VoIP based systems have been gaining popularity internally due their benefits over legacy systems, these include less infrastructure, lower service charge, integrated messaging, global mobility and new tools\cite{website:windows_networking_voip_enterprise}.

Migrating to these VoIP systems enable convergence on a single network plane with IP end-to-end, integration with existing user directory's via Microsoft Active Directory or *nix based LDAP (Lightweight directory access protocol) based which offer centralised logging and access control. But its not just in the administration and back-end where enterprise customers gain advantage, directory integration enables an employees phone line to be diverted to their current location, integration of instant messaging (text), email and calendaring systems and mobility by offering these services to the mobile handset.

Cisco, Avaya and Nortel, are the biggest players in this market with all three company's producing both hardware and software components of an enterprise PBX system. Due to the open nature and interoperability of VoIP, these firms are moving their focus from the hardware to the software offerings. All of the above now offer “unified communication” as opposed to IP telephony, as they try to become the centralised communication hub for today's office. But in this move to a software based solution the market has been opened up to companies like Microsoft who now offer “Office Communications Server”\cite{website:microsoft_enter_uc} and IBM’s Tivoli Netcool Enterprise VoIP Manager.

Telecommunications providers have largely completed the move to digital switching in the mid 1980’s\cite{TelecommsSaN05}, while this technology is not exactly VoIP it acts as an enabler.
