\section{Historical Setting}

In the early days of long distance communication consumers were heavily dependant on telecommunication companies as means of a rapid and reliable communication. However this was soon going to change with the publication of "A Mathematical Theory of Communication" in 1948 by the American mathematician Dr. Claude Shannon, in which he drafted the concept of communicating in binary code. An idea later used to created first digital networks\cite{Shannons_paper}.This paper was going to change the way humans are going to communicate, as digital communications revolution set off to shape the world we are living in today.

Also during this time the American government required a solution to strengthen weak points in existing communications networks. This was a growing concern for American military command, as a nuclear strike could break a communication link between important command and control facilities and severely damage coordination of American troops and prevent them from organising a coordinated response to an attack Twenty years after release of Shannon's paper ARPANET (Advanced Research Projects Agency Network) was developed by USA Dept. of Defence. It was first computer network that later evolved into modern day Internet. Only five years later the first VoIP calls where pioneered on ARPANET as part of research project. However those calls were bounded and restricted to one private network. But the foundation for development of modern VoIP were set\cite{website:Brief_history_voip}.

It was until 1995, the first company was to release VoIP software product. VocalTec a small Israel company released their first commercial product "Internet Phone". It was designed for a domestic PC user utilising commodity hardware\cite{website:Voip_history}. By 1998 espite obstacles including but not limited to the lack of a high speed network infrastructure, according to PriMetrica inc. VocalTec handled 0.2\% of all international phone calls\cite{website:Internet_tel_and_voip}.

In tests with circuit-switched and packet-switched networks, it has been shown that packet-switched networks can carry five to ten times the number of voice calls over the same bandwidth\cite{Voip_corporate}.Better bandwidth utilisation combined with the development growth in the VoIP sector, has attracted numbers of investors and companies who started to work and build on this technology. Cisco and Lucent created a device that could route the VoIP traffic. And as a direct result in 2000 4.3\% of worlds voice traffic was carried over IP and by 2003 this figure had increased to 12.8\%. \cite{website:Voip_history} \cite{website:Internet_tel_and_voip}

It became evident to many that VoIP could be used to improve internal communication within large organisations and bring down cost of telecommunication bills. In 2004 Boeing at that time having 157000 employees and offices in 70 countries decided that all phone calls within company are going to be handled via VoIP. Soon many other companies became to adopt same strategy.


More recently in 2003 the Swedish entrepreneurs Niklas Zennstr�m and the Dane Janus Friis started a company called Skype which was based on generation three peer-to-peer network, which originated in the Kazaa music sharing technology. Currently Skype is one of the leading VoIP provider to businesses and private users.