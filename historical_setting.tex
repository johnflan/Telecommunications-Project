\section{Historical Setting}

Prior to VoIP and due to lack of alternatives, industrial customers were heavily dependants on telecommunication companies as means of rapid and reliable communication. However this was soon going to change as in 1948 American mathematician Dr. Claude Shannon published "A Mathematical Theory of Communication" � paper drafting a concept of communicating in binary code. Idea later used to created first electromechanical switch.\cite{Shannons_paper}

This was a first notable step towards development of VoIP, as it was founding basis of the
entire digital communications revolution, from cell phones to the Internet. 20 years later in 1968 precursor of Internet was born - ARPANET (Advanced Research Projects Agency Network), developed by USA Dept. of Defence.
And only five years later the first VoIP calls where pioneered on ARPANET. However those calls were bounded and restricted to one private network (ARPANET). But the foundation for modern global VoIP were set. \cite{website:Brief_history_voip}

It was until 1995, the first company was to release VoIP software product. VocalTec a small Israel company released their first product "Internet Phone". It  was designed for home PC user utilising sound-card, microphone and speakers. \cite{website:Voip_history}

By 1998 despite obstacles - lack of high speed network infrastructure - VocalTec grabbed stunning 0.2\% of all international phone calls. \cite{website:Internet_tel_and_voip}

When compared with circuit-switched services, IP networks can carry 5 to 10 times the number of voice calls over the same bandwidth.
\cite{Voip_corporate}

This combined with rapid growth of VoIP sector attracted number of big entrepreneurs who started to work and build on this technology. Cisco and Lucent created a device that could route the VoIP traffic. As the result in 2000 4.3\% of worlds voice traffic was done through VoIP.
In 2003 figure increased to 12.8\%. \cite{website:Voip_history} \cite{website:Internet_tel_and_voip}

Soon many companies realised that internal communication costs can be greatly decreased as most of employees have their personal computers and constant Internet connection.
In 2004 Boeing at that time having 157000 employees and offices in 70 countries decided that all phone calls within company are going to be handled via VoIP.


Back in 2003 Swedish entrepreneur Niklas Zennstr�m and the Dane Janus Friis started a new company - Skype.
