\section{Historical Setting}

Prior to VoIP and due to the lack of alternatives, --In the early days of long distance communication--- industrial customers were heavily dependant on telecommunication companies as means of a rapid and reliable communication. However this was soon going to change with the publication of "A Mathematical Theory of Communication" in 1948 by the American mathematician Dr. Claude Shannon, in which he drafted the concept of communicating in binary code. --- DIGITAL NETWORKS--- An idea later used to created first electromechanical switch\cite{Shannons_paper}. This was a first notable step towards development of VoIP, as it was founding basis of the entire digital communications revolution, from cell phones to the Internet. --how human communicated--

--Also during this time the american govenment required a solution to weak points in existing communications networks -- nuclear bomb ---
20 years later in 1968 the precursor of Internet was born - ARPANET (Advanced Research Projects Agency Network), developed by USA Dept. of Defence. --text messages- And only five years later the first VoIP calls where pioneered on ARPANET. However those calls were bounded and restricted to one private network. But the foundation for modern VoIP were set\cite{website:Brief_history_voip}.

It was until 1995, the first company was to release VoIP software product. VocalTec a small Israel company released their first commercial product "Internet Phone". It was designed for a domestic PC user utilising commodity hardware\cite{website:Voip_history}. By 1998 despite obstacles including but not limited to the lack of a high speed network infrastructure, --according to-- VocalTec handled 0.2\% of all international phone calls\cite{website:Internet_tel_and_voip}.

---In tests with circuit-switched and packet-switched networks, it has been shown that packet-switched networks can carry five to ten times the .....---
When compared with circuit-switched services, IP networks can carry 5 to 10 times the number of voice calls over the same bandwidth\cite{Voip_corporate}.This better bandwidth utilisation combined with the development growth in the VoIP sector, has attracted numbers of --- ---big entrepreneurs who started to work and build on this technology. Cisco and Lucent created a device that could route the VoIP traffic. And as a direct result in 2000 4.3\% of worlds voice traffic was carried over IP and by 2003 this figure had increased to 12.8\%. \cite{website:Voip_history} \cite{website:Internet_tel_and_voip}

--It became evidant to many....--Soon many companies realised that internal communication costs can be greatly decreased as most of employees have their personal computers and constant Internet connection.
In 2004 Boeing at that time having 157000 employees and offices in 70 countries decided that all phone calls within company are going to be handled via VoIP.


Back in 2003 Swedish entrepreneur Niklas Zennstr�m and the Dane Janus Friis started a company - Skype -came from kazaa- . ebay Currently Skype is the leading VoIP provider to businesses and private users.
